\documentclass[11pt]{article}
\usepackage{geometry}
\geometry{a4paper,left=3cm,right=3cm, top=3cm, bottom=3cm}
\usepackage[parfill]{parskip}    % Activate to begin paragraphs with an empty line
\usepackage[T1]{fontenc}
\usepackage{graphicx}
\usepackage{tabularx}
\usepackage{fancyhdr}
\usepackage{lmodern}
\usepackage{amssymb}
\usepackage{epstopdf}
\usepackage{subfig}
\usepackage{hyperref}
\usepackage{mathtools}
\usepackage{listings}
\usepackage[table]{xcolor}

\newcommand{\groupNr}{49}
\newcommand{\assignmentNr}{3}

\pagestyle{fancy}
\fancyhf{}
\lfoot{SWA2011}
\cfoot{\thepage}
\rfoot{Group: \groupNr}
\renewcommand{\footrulewidth}{0.0pt}
\renewcommand{\headrulewidth}{0.0pt}

\begin{document}
{
	\begin{figure}[h]
	\hspace{-1cm}
	\includegraphics[height=23mm, width=170mm]{dsg-logo.png}
	\end{figure}

	\vspace{25mm}
	\centering
	{ \sffamily \Huge Assignment \assignmentNr } \\ \vspace{1mm}
	Software Architekturen \\ SS2010 \\ \vspace{10mm}
}

\noindent Group: \groupNr
\begin{itemize}
\item Patrick Marschik, Mat. Nr.: 0625039, Stud. Kz.: 066 933
\item Martin Schwengerer, Mat. Nr.: 0625209, Stud. Kz.: 066 937
\item Michael V\"ogler, Mat. Nr.: 0625617, Stud. Kz.: 066 937
\item Matthias Rauch, Mat. Nr.: 0626140, Stud. Kz.: 033 534
\item Benjamin Bachhuber, Mat. Nr.: 1028430, Stud. Kz.: 066 933
\end{itemize}

\newpage
\tableofcontents

\newpage

\section{Technical Specification}

\begin{table}[h]
	\begin{tabularx}{\textwidth}{| X | X |}
	\hline
	\multicolumn{2}{|c|}{\textbf{Servers}} \\
	\hline
	\cellcolor[gray]{0.9}
	Application Server & Tomcat\\
	\hline
	\cellcolor[gray]{0.9}
	DBMS & PostgreSQL 8.4\\
	\hline
	\cellcolor[gray]{0.9}
	Cache & EhCache\\
	\hline
	\end{tabularx}
	\caption{Server technology}
	\label{server_tech}
\end{table}

\begin{table}[h]
	\begin{tabularx}{\textwidth}{| X | X |}
	\hline
	\multicolumn{2}{|c|}{\textbf{Frameworks}} \\
	\hline
	\cellcolor[gray]{0.9}
	Dependency Injection & Spring\\
	\hline
	\cellcolor[gray]{0.9}
	OR-Mapping & Hibernate\\
	\hline
	\cellcolor[gray]{0.9}
	Cache & EhCache\\
	\hline
	\cellcolor[gray]{0.9}
	MVC-Framework & Spring MVC \& J-Query\\
	\hline
	\cellcolor[gray]{0.9}
	Messaging & Spring Integration\\
	\hline
	\end{tabularx}
	\caption{Frameworks}
	\label{frameworks}
\end{table}


\begin{table}[h]
	\begin{tabularx}{\textwidth}{| X | X |}
	\hline
	\multicolumn{2}{|c|}{\textbf{Tools}} \\
	\hline
	\cellcolor[gray]{0.9}
	Build System & Maven\\
	\hline
	\cellcolor[gray]{0.9}
	IDE & IntelliJ\\
	\hline
	\cellcolor[gray]{0.9}
	Version Control System & Git\\
	\hline
	\end{tabularx}
	\caption{Tools}
	\label{tools}
\end{table}

\section{Prototype Architecture}

\subsection{(Entity) Class Diagram}

The entities of the system are distributed over three different persistence units. 

\begin{description}
\item[Map domain (see Figure \ref{fig:map_cd})]
	The map domain describes all the classes that are needed to play on a specific map. The maps are created by the system and not the users. Maps consist of tiles. Each tile can have a base which consists of 32 squares. Upgrading buildings and sending troops takes some time. Creating troops and running buildings may cost some resources.
\item[Message domain (see Figure \ref{fig:message_cd})]
	The message domain consists of all classes that are persisted due to messaging. Message dates are stored in UTC.
\item[User domain (see Figure \ref{fig:user_cd})]
	The user domain consists of all classes that are persisted by the user management component. The User embeds address information and stores the difference to UTC to ensure messages are sent in right order. 
\end{description}

\subsection{Component Diagram}

The component diagram (see Figure \ref{fig:component}) illustrates the interaction between the components. The Map Component encapsulates the logic needed to play on a specific map. It offers logic to navigate on a map and perform actions. The actual processing of these actions is performed by the Map Processing component, which also provides an environment for periodic task like statistic calculation. The User Management Component offers interfaces for registration/deletion of accounts. It also exports a TokenService, which is used for authentication. Finally the message component offers an interface for message dispatch and receipt. All Storage components can be configured to use application level caching.

\subsection{Deployment Diagram}

The deployment diagram (see Figure \ref{fig:deployment})  shows three main sites where artifacts have to be deployed.
\begin{enumerate}
\item There exists a cluster hosting Map Controller and the corresponding database for each map in the game. The cluster itself can be divided into several nodes that can server requests for this map. A load balancer is responsible for the request routing. Since the data for each map can be separated from data for other maps replication must not be used here. As shown in the diagram each map node can have multiple Map Processing Servers, which access the same database. For smaller maps, the swag49.web.processing could be deployed to the Map Server instead.
\item A similar cluster hosting the User Management component runs the User Management component and  serves authentication requests. Again the load-balancer is the access point for clients.
\item Another cluster is responsible for processing messaging requests. As denoted in D-1 there exists a message queue for each node of the cluster to provide reliable messaging. A redundant network of SMTP servers ensures availability of e-mail notifications.
\end{enumerate}

\subsubsection{Uptime Calculation}

This redundancy of services is necessary on the one hand to fulfill the performance requirements and provide a scalable architecture. On the other hand it results in a fail safe system as the following calculation shows:

\begin{itemize}
\item $P(nodeFails) = 0.1$
\end{itemize}

As the requirements specification denotes we can assume that each node has a failure probability of $10\%$.

A cluster consists of n nodes. Since the cluster is available when at least one node is available we can set failure possibility of a cluster.

\begin{itemize}
\item $P(clusterFails) = 0.1^n$
\end{itemize}

When we consider that we have three cluster in our system, we get the formula for the overall system fail rate:

\begin{itemize}
\item $P(clusterFails) = 3 *0.1^n$
\end{itemize}

To calculate availability we have to consider the inverse probability:

\begin{itemize}
\item $P(systemAvailable) = 1 - 3 *0.1^n$
\end{itemize}

If we solve the inequality

\begin{itemize}
\item $1 - 3 *0.1^n \ge 0.99999$
\end{itemize}

we receive

\begin{itemize}
\item $n \ge 4.47712 \implies n \ge 5$
\end{itemize}

That means if that we reach an availability of $99,99\%$ if we have at least 5 nodes in each cluster. To fulfill security requirements the communication between server and client is always encrypted.

\newpage

\subsection{Architectural Decisions}

\subsubsection{D-1}

\begin{table}[h] \small
	\begin{tabularx}{\textwidth}{ | l | X |}
    	\hline
	\cellcolor[gray]{0.9}
    	\textbf{Issue} & There can be lots of simultaneous messages and not all of them can be handled by the database and mail servers directly. \\
	\hline
	\cellcolor[gray]{0.9}
	\textbf{Decision} & Use an asynchronous message queue as buffer for sent messages. The middleware has to offer a queue for each Notification node. \\
	\hline
	\cellcolor[gray]{0.9}
	\textbf{Group} & Component Interaction \\
	\hline
	\cellcolor[gray]{0.9}
	\textbf{Assumptions} &
		\begin{itemize}
		\item Lots of simultaneous messages
		\item Not all of them can be handled directly by processing nodes
		\end{itemize} \\
	\hline
	\cellcolor[gray]{0.9}
	\textbf{Constraints} & - \\
	\hline
	\cellcolor[gray]{0.9}
	\textbf{Positions} &
		\begin{itemize}
		\item Directly send/store messages at the mail server resp. database server using explicit invocation.
		\item Use one single message queue for all e-mails and use one single for all internal messages.
		\end{itemize} \\
	\hline
	\cellcolor[gray]{0.9}
	\textbf{Argument} & Message queues buffer messages to ensure the system can cope with load peeks. Since processing of the message is delayed a call to the Notification Component would take a much smaller amount of time. I also decided to use a queue for each processing node, since a central queue would cause a single-point of failure. \\
	\hline
	\cellcolor[gray]{0.9}
	\textbf{Implications} & The middleware must be chosen appropriately to support asynchronous message queues. \\
	\hline
	\cellcolor[gray]{0.9}
	\textbf{Related decisions} & - \\
	\hline
	\cellcolor[gray]{0.9}
	\textbf{Related requirements} &
		\begin{itemize}
		\item There can be lots of simultaneous messages and not all of them can be handled by the database and mail servers directly.
		\item Make sure that notifications are reliable and do not simply rely on the database or, even worse, the mail server.
		\end{itemize}\\
	\hline
	\cellcolor[gray]{0.9}
	\textbf{Related artifacts} & requirements specification, component diagram, deployment diagram \\
	\hline
	\cellcolor[gray]{0.9}
	\textbf{Related principles} & -\\
	\hline
	\end{tabularx}
	\caption{Design decision - D-1}
	\label{dec:D1}
\end{table}

\newpage

\subsubsection{D-2}

\begin{table}[h] \small
	\begin{tabularx}{\textwidth}{ | l | X |}
    	\hline
	\cellcolor[gray]{0.9}
    	\textbf{Issue} & A complex system should be divided into components to enforce separation of concerns and provide reusability and modifiability. \\
	\hline
	\cellcolor[gray]{0.9}
	\textbf{Decision} & Structure the architecture into layers, s.t. higher layers depend on lower layers. \\
	\hline
	\cellcolor[gray]{0.9}
	\textbf{Group} & Component Interaction \\
	\hline
	\cellcolor[gray]{0.9}
	\textbf{Assumptions} &
		\begin{itemize}
		\item The functionality can be grouped into components.
		\item The components can define interfaces to make their functionality externally available.
		\end{itemize}\\
	\hline
	\cellcolor[gray]{0.9}
	\textbf{Constraints} & - \\
	\hline
	\cellcolor[gray]{0.9}
	\textbf{Positions} &
		\begin{itemize}
		\item Use strong coupling between components.
		\item Use a monolithic design.
		\end{itemize}\\
	\hline
	\cellcolor[gray]{0.9}
	\textbf{Argument} & Some low-level parts of the system (e.g. Persistence, Access Control) are used by many higher-level parts. Strong coupling between components would restrain us concerning modifications be done in future, since we could not exchange components. A monolithic design on the other hand restrains concerning distributability of the components. \\
	\hline
	\cellcolor[gray]{0.9}
	\textbf{Implications} & The architecture should be grouped into the following layers (from high to low):
		\begin{enumerate}
		\item Presentation
		\item Business Logic (Maps, Statistics, User Management)
		\item Cache
		\item Persistence
		\end{enumerate}\\
	\hline
	\cellcolor[gray]{0.9}
	\textbf{Related decisions} & - \\
	\hline
	\cellcolor[gray]{0.9}
	\textbf{Related requirements} & -\\
	\hline
	\cellcolor[gray]{0.9}
	\textbf{Related artifacts} & Component diagram\\
	\hline
	\cellcolor[gray]{0.9}
	\textbf{Related principles} & Dependency Injection\\
	\hline
	\end{tabularx}
	\caption{Design decision - D-2}
	\label{dec:D2}
\end{table}

\newpage

\subsubsection{D-3}

\begin{table}[h] \small
	\begin{tabularx}{\textwidth}{ | l | X |}
    	\hline
	\cellcolor[gray]{0.9}
    	\textbf{Issue} & Important actions have to be logged. It is also necessary to monitor system performance. \\
	\hline
	\cellcolor[gray]{0.9}
	\textbf{Decision} & Use a logging component that writes log message to a database. Other components use this logging component. \\
	\hline
	\cellcolor[gray]{0.9}
	\textbf{Group} & Adaption \\
	\hline
	\cellcolor[gray]{0.9}
	\textbf{Assumptions} & - \\
	\hline
	\cellcolor[gray]{0.9}
	\textbf{Constraints} & - \\
	\hline
	\cellcolor[gray]{0.9}
	\textbf{Positions} &
		\begin{itemize}
		\item Use interceptors for logging.
		\item Use profiling tools to monitor application and database.
		\end{itemize}\\
	\hline
	\cellcolor[gray]{0.9}
	\textbf{Argument} & Interceptors have the advantage that the code doesn't get polluted with logging statements. However it is also possible to use AOP aspects to introduce logging at a later stage. This provides more flexibility since AOP doesn't need hooks to be plugged in.  Compared to profiling tools, modern logging libraries are much cheaper. Besides that profiling tools consume more resources. \\
	\hline
	\cellcolor[gray]{0.9}
	\textbf{Implications} & Logging should happen in the following scenarios:
		\begin{itemize}
		\item user login/logout
		\item action start/end
		\item database access
		\item notification sending/receiving
		\item complex calculations (map generation, attacks, etc.)
		\end{itemize}\\
	\hline
	\cellcolor[gray]{0.9}
	\textbf{Related decisions} & - \\
	\hline
	\cellcolor[gray]{0.9}
	\textbf{Related requirements} & Every important action in the system has to be logged. There should be a user ranking which can be seen by every user: user with most points, richest user, strongest troop type, and so on. Also try to monitor some aspects of the system performance (e.g., average processing times, resource usage) and the system configuration itself (e.g., currently active nodes). Try to keep this information as up-to-date as possible, but do not create it directly from live data.\\
	\hline
	\cellcolor[gray]{0.9}
	\textbf{Related artifacts} & Requirements specification\\
	\hline
	\cellcolor[gray]{0.9}
	\textbf{Related principles} & -\\
	\hline
	\end{tabularx}
	\caption{Design decision - D-3}
	\label{dec:D3}
\end{table}

\newpage

\subsubsection{D-4}

\begin{table}[h] \small
	\begin{tabularx}{\textwidth}{ | l | X |}
    	\hline
	\cellcolor[gray]{0.9}
    	\textbf{Issue} &  Actions take some time. The execution is delayed.\\
	\hline
	\cellcolor[gray]{0.9}
	\textbf{Decision} & Use a job scheduling approach. \\
	\hline
	\cellcolor[gray]{0.9}
	\textbf{Group} & Business Logic \\
	\hline
	\cellcolor[gray]{0.9}
	\textbf{Assumptions} & - \\
	\hline
	\cellcolor[gray]{0.9}
	\textbf{Constraints} & - \\
	\hline
	\cellcolor[gray]{0.9}
	\textbf{Positions} &
		\begin{itemize}
		\item Execute actions when they are performed and mark the result as inactive.
		\end{itemize}\\
	\hline
	\cellcolor[gray]{0.9}
	\textbf{Argument} & The job approach has the advantage that the result of the action is visible as soon as the action finishes. So there is no need to store temporary action results and we don't have to care about issues that arise when actions should be canceled. In this case the scheduled job would simply be deleted. Existing job scheduling libraries also offer the possibility to store the job and trigger data in database, which would be needed in a distributed environment.  \\
	\hline
	\cellcolor[gray]{0.9}
	\textbf{Implications} & The job scheduling library must support clustered execution of jobs, since action execution is one of the most performance crucial parts of the whole system. Therefore this task can't be handled by a single server.\\
	\hline
	\cellcolor[gray]{0.9}
	\textbf{Related decisions} & D-5 \\
	\hline
	\cellcolor[gray]{0.9}
	\textbf{Related requirements} & Performance must be consistent. It is not acceptable for a user to have to wait more than two or three seconds when submitting a post or loading a page. So think about a good strategy how to scale all parts of the application.\\
	\hline
	\cellcolor[gray]{0.9}
	\textbf{Related artifacts} & Requirements specification\\
	\hline
	\cellcolor[gray]{0.9}
	\textbf{Related principles} & -\\
	\hline
	\end{tabularx}
	\caption{Design decision - D-4}
	\label{dec:D4}
\end{table}

\newpage

\subsubsection{D-5}

\begin{table}[h] \small
	\begin{tabularx}{\textwidth}{ | l | X |}
    	\hline
	\cellcolor[gray]{0.9}
    	\textbf{Issue} & Performance is crucial. The system needs to handle 1000's of concurrent users. \\
	\hline
	\cellcolor[gray]{0.9}
	\textbf{Decision} & It should be possible to partition the system horizontally for each map. That means that each map should have a denoted server (or server farm). Besides that the storage components are configured to use a cache to minimize database roundtrips. \\
	\hline
	\cellcolor[gray]{0.9}
	\textbf{Group} & Performance \\
	\hline
	\cellcolor[gray]{0.9}
	\textbf{Assumptions} & The system can be partitioned horizontally for each map. \\
	\hline
	\cellcolor[gray]{0.9}
	\textbf{Constraints} & - \\
	\hline
	\cellcolor[gray]{0.9}
	\textbf{Positions} &
		\begin{itemize}
		\item Set up a single server that is strong enough to handle 1000's of users.
		\item Set up many nodes that mirror the whole database.
		\end{itemize}\\
	\hline
	\cellcolor[gray]{0.9}
	\textbf{Argument} & Distributing the application logic over more than one server is a good idea for scenarios where performance and availability are key requirements. Concerning availability clustering has the benefit that we don't have a single point of failure. Besides that in our concrete scenario, the map is the perfect choice for a partition criteria, since there are no cross-map operations possible by requirements specification. This results in less replication overhead. \\
	\hline
	\cellcolor[gray]{0.9}
	\textbf{Implications} & Load balancing \\
	\hline
	\cellcolor[gray]{0.9}
	\textbf{Related decisions} & D-2 (cache layer) \\
	\hline
	\cellcolor[gray]{0.9}
	\textbf{Related requirements} & Performance must be consistent. It is not acceptable for a user to have to wait more than two or three seconds when submitting a post or loading a page. So think about a good strategy how to scale all parts of the application.\\
	\hline
	\cellcolor[gray]{0.9}
	\textbf{Related artifacts} & Requirements specification, deployment diagram\\
	\hline
	\cellcolor[gray]{0.9}
	\textbf{Related principles} & -\\
	\hline
	\end{tabularx}
	\caption{Design decision - D-5}
	\label{dec:D5}
\end{table}

\newpage

\section{Prototype Installation Guidelines}

\subsection{Requirements}

\begin{itemize}
\item Tomcat 6 or 7 running on \texttt{localhost:8080/}
\item PostgreSQL 8.4 running on \texttt{localhost:5432/}
	\begin{itemize}
	\item Database: \texttt{swa}
		\begin{itemize}
		\item accessible by user: \texttt{swa}
		\item password for user: \texttt{swa11}
		\end{itemize}
	\end{itemize}
\item Ant and Maven 2
\item The environment variable \texttt{TOMCAT\_HOME} must be set to the installation directory of tomcat.
\end{itemize}

\subsection{Compiling and Deploying}

To test the demo application you have to follow these steps:

\begin{enumerate}
\item Create an empty database as denoted above.
\item Execute \texttt{ant deploy} to deploy the application to tomcat.
\item Execute \texttt{ant setup} to prepare the database.
\item Start tomcat. You will have to set the max. memory of the JVM at least to 512 MB.
\end{enumerate}

\subsection{Entry Point}

The entry point to play \textbf{SWAG} is \href{http:localhost:8080/user/swag/user/}{http:localhost:8080/user/swag/user/}. There you can register an user, login and choose a map to play on.

\newpage

\pagenumbering{Roman}
\appendix
\section{Figures}

\begin{figure}[h]
\center
\includegraphics[angle=90, scale=0.3]{diagrams/map.png}
\caption{SWAG - Map Class Diagram}
\label{fig:map_cd}
\end{figure}

\begin{figure}[h]
\center
\includegraphics[scale=1]{diagrams/user.png}
\caption{SWAG - User Class Diagram}
\label{fig:user_cd}
\end{figure}

\begin{figure}[h]
\center
\includegraphics[scale=1]{diagrams/message.png}
\caption{SWAG - Message Class Diagram}
\label{fig:message_cd}
\end{figure}

\begin{figure}[h]
\center
\includegraphics[scale=0.7]{diagrams/component.png}
\caption{SWAG - Component Diagram}
\label{fig:component}
\end{figure}

\begin{figure}[h]
\center
\includegraphics[angle=90, scale=0.6]{diagrams/deployment.png}
\caption{SWAG - Deployment Diagram}
\label{fig:deployment}
\end{figure}

 \end{document}
